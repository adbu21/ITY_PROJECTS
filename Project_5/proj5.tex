\documentclass[11pt, hyperref={unicode}]{beamer}
\usetheme{Boadilla}
\usecolortheme{default}
\usepackage[utf8]{inputenc}
\usepackage[czech]{babel}
\usepackage{graphicx} % beamer loads graphicx by itself (+ enumerate,..)
\usepackage{xcolor}
\usepackage[linesnumbered, ruled, longend]{algorithm2e}

\title{Řadící algoritmy - Bubble sort}
\subtitle{Typografie a publikování - 5. projekt}
\author{Adam Buchta (xbucht29)}
\date{\today}

\begin{document}

    \begin{frame}
        \titlepage
    \end{frame}

    \begin{frame}
        \tableofcontents
    \end{frame}

    \begin{frame}
        \centering
        {\color{blue!65!black!90!white}\Huge{Základní definice}}
    \end{frame}

    \section{Základní definice}
    \begin{frame}
        \frametitle{Základní definice}
        \begin{itemize}
          \item Bubble sort (či bublinkové řazení) je jednoduchý řadící algoritmus se \alert{složitostí} $O(n^2)$.
          \item Složitost udává rychlost algoritmu vzhledem k množině vstupních dat.
          \item Bubble sort je nejpomalejším řadícím algoritmem.
        \end{itemize}
    \end{frame}

    \section{Princip}
    \begin{frame}
          \begin{figure}
            \centering
            {\color{blue!65!black!90!white}\huge{Princip}}
                                \\~\
                                \\~\

            \includegraphics[width=230px]{BubbleSort.jpg}
          \end{figure}
    \end{frame}

    \begin{frame}
        \frametitle{Princip}
        \begin{itemize}
          \item Bublinky s menší hodnotou jsou lehčí a stoupají proto ve vodě rychleji.
          \item Bubble sort porovnává každý prvek se svým následníkem. Je-li tento prvek větší/menší, zamění je.
          \item Toto provede pro celou posloupnost \alert{$n$} prvků
          \item Celý postup musíme aplikovat \alert{$n-1$} krát pro jistotu seřazení prvků.
        \end{itemize}
    \end{frame}

    \begin{frame}
        \centering
        {\color{blue!65!black!90!white}\Huge{Pseudokód}}
    \end{frame}

    \section{Pseudokód}

    \begin{frame}
        \frametitle{Pseudokód}
        \textbf{Popis algoritmu v krocích}
        \pause
        \begin{enumerate}
          \item Posloupnost rozdělíme na setříděnou a nesetříděnou část. {\color{blue!75!white!40!gray}(setříděná část je prázdná)}
              \pause
          \item Postupně porovnáme všechny sousední prvky v nesetříděné části a prohodíme je, pokud nejsou v požadovaném pořadí.
              \pause
          \item Krok 2 opakujeme, dokud nesetříděná část obsahuje více než jeden prvek.
        \end{enumerate}
    \end{frame}

    \begin{frame}

        \begin{algorithm}[H]
            \For{$i$ in 1 $->$ a.length - 1}{
                \For{$j$ in 1 $->$ a.length - $i$ - 1}{
                    \If{$a[j] < a[j+1]$}{
                        switch($a[j], a[j+1]$)}}}

            \caption{Pseudocode}
        \end{algorithm}
    \end{frame}

    \begin{frame}
        \centering
        {\color{blue!65!black!90!white}\Huge{Bubble sort s výhodou}}
    \end{frame}

    \section{Bubble sort s výhodou}

    \begin{frame}
        \frametitle{Bubble sort s výhodou}
        \begin{itemize}
          \item Navíc zavedena \alert{proměnná} sledující výměnu prvků.
          \item Při alespoň jedné výměně se v celém poli opakuje porovnávání ještě jednou.
          \item Tato metoda je jedna z nejrychlejších při porovnávání již seřazeného pole.
          \item Tohoto můžeme využít při zjišťování, zda je pole již seřazené.
        \end{itemize}
    \end{frame}

    \begin{frame}
        \frametitle{Použité zdroje}
        \begin{thebibliography}{99}
          \bibitem[Algo]{Algo} Algoritmy.net - Bubble sort
            \url{https://www.algoritmy.net/article/3/Bubble-sort}
          \bibitem[Mendel]{Mendel} Mendel University in Brno - Řadící algoritmus
              
            \url{https://is.mendelu.cz/eknihovna/opory/zobraz_cast.pl?cast=65555}
          \bibitem[Java]{Java} Java Algoritmy - Bubble sort
            \url{http://javaalgoritmy.wz.cz/bubble.htm}
          \bibitem[CS]{CS} CS Unplugged - Sorting Algorithms
            \url{https://classic.csunplugged.org/sorting-algorithms/}
        \end{thebibliography}
    \end{frame}




\end{document} 